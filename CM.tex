\documentclass[12pt,a4paper]{report}
\usepackage[utf8]{inputenc}
\usepackage[french]{babel}
\usepackage[T1]{fontenc}
\usepackage{amsmath}
\usepackage{amsfonts}
\usepackage{amssymb}
\usepackage{graphicx}
\author{Malo Kerebel}

\usepackage{systeme}

\usepackage[mathscr]{euscript}

\usepackage{hyperref}
\hypersetup{
    colorlinks,
    citecolor=black,
    filecolor=black,
    linkcolor=black, urlcolor=black
}

\newcommand{\Ens}[1]{\mathbb{#1}}
\newcommand{\ens}[1]{\mathbb{#1}}
\newcommand{\fphi}{\quad \forall \varphi \in \mathcal{D}}
\newcommand{\D}{\ensuremath{\mathcal{D}}}

\input cyracc.def
\font\tencyr=wncysc10
\def\cyr{\tencyr\cyracc}
\def\dc{\mbox{\cyr SH}}


\begin{document}

\begin{titlepage}

\centering{
	
	{\scshape\LARGE Université de Bretagne Occidentale \par}
	\vspace{1cm}
	{\scshape\Large Note de cours\par}
	\vspace{1.5cm}
	{\huge\bfseries Outils fondamentaux pour la physique \unskip\strut\par}
	\vspace{2cm}
	{\Large\itshape Malo Kerebel \par}
	\vfill
	Cours par\par
	Pascal \textsc{Rivière}

	\vfill

% Bottom of the page
	{\large Semestre 6, année 2020-2021 \par}
}
\end{titlepage}

\tableofcontents

\chapter{Distributions et fonctions test}
\begin{center}
\textbf{CC1 (2021-01-13): Distributions}
\end{center}

\section{Introduction}

Dernier outils en maths pour la physique. Introduit pour résoudre des problèmes physiques que l'on ne pouvait pas traiter avec les fonction, eg:
\begin{enumerate}
	\item Impulsion
	\item Electromagnétisme
	\item Mécanique quantique
\end{enumerate}

\textbf{Exemple 1}, une bille fait un choc dur avec une bille au repos qui ne recevait aucune énergie avant le choc et après le choc, elle reçoit donc une énergie fini en un temps nul

On représente l'énergie par une fonction f telle que:
\[
	\int_{\mathbb{R}} f(t) dt = E \text{fini}
\]
\[
	f(t) = 0 \forall t \in \mathbb{R}^*
\]
La théorie de l'intégration nous dirait que l'intégrale est nulle

On peut tenter un passage à la limite :\\
\(f_n(t) \systeme{n \text{si } t \in [ \tfrac{-1}{2n} \tfrac{1}{2n} ],0 \text{ sinon} }\)

\textbf{Exemple 2} Potentiel électrostatique.
Distribution de charge continue : V(r) = k 
\[
	\int\int\int_{\mathbb{R}^3} \dfrac{\rho(r')}{\vert r- r'\vert} dv' \text{ avec } k = \dfrac{1}{4\pi\varepsilon_0} 
\]

Distribution de charge ponctuelle :
\[
	V(r) = k\dfrac{Q}{\vert r - a \vert} = k\dfrac{1}{\vert r - a \vert}
\]
Pour faire le lien entre les deux il faudrait \(\rho\) telle que \(\int\int\int \rho = Q\) mais avec \(\rho = 0\) sauf en a

Dirac a donc créer la distribution de dirac \(\delta_a\)
\[
	\int_R \delta_a \varphi dx =_{notation} \langle \delta_a, \varphi \rangle =_{definition} \varphi(a)
\]

Toute la théorie des distribution va donc venir de la définitons de ces fonction $\varphi$, les fonctions test.

\section{L'espace \D des fonctions test}

\paragraph{Définition 1.1.} L'espace \D\\
Ces fonctions sont indéfiniment dérivable $C^\infty$ et à support borné, elle sont nulles en dehors d'un intervalle [a,b] ensuite elles sont lisses entre a et b

\paragraph{Définition 1.2.} Support d'une fonction\\
Si \(\varphi\) est une fonction, son support est le plus petit ensemble fermé en dehors duquel elle est nulle. Ça se note supp(\(\varphi\)).

Puisque \(\varphi\) est \(C^{\infty}\) on doit avoir \(\varphi^{(n)}(a) = \varphi^{(n)}(b) = 0, \forall n \in \mathbb{N}\)

Exemples : \par
- La fonction dite de Schwartz:
\[
	\varphi_s (x) = e^{\frac{1}{x^2-1} }\text{ si } \vert x \vert < 1
\]

On doit vérifier que les dérivé sont bien nulle en -1 ou 1 (la fonction est paire donc ça ne change rien)

On peut toujours trouver des fonction \(\varphi \in \D\) de support [c,d] et telle que \(\varphi = 1\) sur [a,b], où \(c < a < b < d \in \mathbb{R}\)

De même si au lieu de \(\varphi = 1\) on a \(\varphi = f\), où \(f \in C^{\infty}\)

Ex :  \[\text{1) } \psi_a (x) = \varphi_s \left(\frac{x}{a}\right) \times \frac{1}{\int_R \varphi_s \left(\frac{x}{a}\right) dx}\]
\[
	\text{2) }h = \psi_a (x) * \chi_{[0,L]}
\]

Pour \(\varphi = f\) on multiplus juste la fonction pour \(\varphi = 1\) par f.

\paragraph{Définition 1.3.} Convergence dans \(\D\)\\
\(\phi_n \rightarrow \phi\) dans \D si et seulement si il existe K inclus dans R borné, tel que le support de \(\phi_n\) est inclus dans \(\mathbb{K}\) \(\forall n\) et le support de \(\phi\) est inclus dans \(\mathbb{K}\) ainsi que \(\varphi_n^{(\alpha} \rightarrow \varphi^{(\alpha}\) uniformement \(\forall \alpha \in \mathbb{N}\)

\paragraph{Propriétés 1.1.}
\begin{enumerate}
	\item Si \(\varphi \in \D\) et f est une fonction à support borné, alors : \(\varphi * f \in \D\)
	\item \(\forall f L^1_loc(\mathbb{R})\) (localement sommable, c'est à dire sommable sur tout intervalle borné de \(\mathbb{R}\)), \(\forall \varphi \in \D\) :\quad \(f\cdot \varphi \in L^1(\mathbb{R})\)
	\item Si f et g sont des fonctions \(L^1_{loc}\):
	\[
		f =_{pp} g \Leftrightarrow \int_{\mathbb{R}} f(x) \varphi(x) dx = \int_{\mathbb{R}} g(x)  \varphi(x) dx \forall \varphi \in \D
	\]
	Ce qui revient à dire que toute fonction \(d \in L^1_{loc}\)
\end{enumerate}

\section{L'espace \D' des distributions}

\paragraph{Définition 1.4.} Distributions
Une distributions T est une application linéaire continue de \(\mathbb{D}\) dans \(\mathbb{R}\).
On note \D' l'ensemble des distributions. \D' est aussi appelé espace dual de \D.

\textbf{Exemples :}

\paragraph{1.} Distribution \(\delta\)
\paragraph{Définition 1.5.} Distrirbution de Dirac en 0 :

On note \(\delta\) la distribution de dirac en 0 définie par :
\[
	\langle \delta, \varphi \rangle = \varphi(0) \forall \varphi \in \D
\]

\begin{enumerate}
	\item On a bien une application de \(\D \text{ dans } \mathbb{R}\)
	\item Linéarité : \(\langle \delta, \alpha_1 \varphi_1 + 	\alpha_2 \varphi_2 \rangle = \alpha_1 \langle \delta, \varphi_1 \rangle + \alpha_2 \langle \delta, \varphi_2 \rangle\)
	\item Continuité \(\varphi_n \rightarrow \varphi\)\\
	\(\langle \delta, \varphi_n \rangle = \varphi_n (0) \rightarrow \varphi(0) = \langle \delta, \varphi \rangle\) donc c'est bien continu 
\end{enumerate}

\paragraph{2.} L'application suivante est aussi une distribution de \D' :
\[
	T : \varphi \rightarrow \int_{\Ens{R}} \varphi(x) dx
\]
\begin{enumerate}
	\item L'intégrale donne bien un réelle donc c'est bien une application de \(\D\) dans \(\Ens{R}\)
	\item L'intégrale est linéaire donc cette application est linéaire aussi
	\item Soit \(\varphi_n \rightarrow \varphi\) dans \D
	\[
		\lim_{n \rightarrow +\infty} \langle T, \varphi_n \rangle = \lim_{n \rightarrow +\infty} \int_{\ens{R}} \varphi_n (x) dx = \int_{\ens{R}} \lim_{n \rightarrow +\infty} \varphi (x) dx = \int_{\ens{R}} \varphi(x) dx
	\]
	Donc T est continue
\end{enumerate}

\begin{center}
\textbf{CM2 (2021-01-14): Distributions}
\end{center}

\paragraph{1.6.} Somme et multiplication par un scalaire :

Si \(T_1, T_2 \in \D\) :
\[
	\langle T_1 + T_2, \varphi \rangle = \langle T_1, \varphi \rangle + \langle T_2, \varphi \rangle \quad \forall \varphi \in \D
\]Similaire au fonctionnement d'une intégrale

Si \(T \in \D' \text{ et } \alpha \in \ens{R}\)
\[
	\langle \alpha T, \varphi \rangle = \alpha \langle T, \varphi \rangle
\]

\paragraph{Exemple :} \(\delta + \delta = 2\delta\)
\[
	\langle \delta + \delta, \varphi \rangle = \langle \delta, \varphi \rangle + \langle \delta, \varphi \rangle = 2 \langle \delta, \varphi \rangle =  \langle 2\delta, \varphi \rangle \quad \forall \varphi \in \D
\]
\[
	\langle \delta + \delta, \varphi \rangle = 2 \varphi(0)
\]
\paragraph{Définition 1.7.} Distribution nulle
\[
	T = 0 \text{ dans } \D' \Leftrightarrow \langle \alpha T, \varphi \rangle = 0\quad \forall \varphi \in \D
\]

\paragraph{Définition 1.8.} Égalité dans \D'
\[
	T_1 = T_2 \text{ dans } \D' \Leftrightarrow T_1 - T_2 = 0 \text{ dans } \D'
\]
C'est à dire :
\[
	T_1 = T_2 \text{ dans } \D' \Leftrightarrow \langle \alpha T_1 - T_2, \varphi \rangle = 0\quad \forall \varphi \in \D
\]

\paragraph{Théorème 1.2.} (\(\D',+, \cdot, 0\) est un espace vectoriel de dimension infinie

\paragraph{Définition 1.9.} Distributions régulières et singulières
\begin{enumerate}
	\item On appelle distribution régulière un distribution que l'on peut définir à partir d'une fonction f de façon :
	\[
		\langle T_f, \varphi \rangle = \int_{\ens{R}} f(x) \varphi{x} dx \quad \forall \varphi \in \D
	\]
	C'est équivalent au produit scalaire de f par \(\varphi\), on notera \(T_f\) ou \([f]\)
	\item Toutes les autres distributions sont singulières
\end{enumerate}

La distribution \(\delta\) est par exemple singulière, tandis que \(T : \varphi \rightarrow \int_{\Ens{R}} \varphi(x) dx\) est une distribution régulière

\paragraph{Remarques :}
\begin{enumerate}
	\item Toute fonction de \(L^1_{loc}(\ens{R}\) définit une distribution (régulière)
	\item Par abus de langage on pourra dire "Soit \(f \in L^1_{loc} (\ens{R})\) une distribution"
	\item Inversement, par abus on pourra dire aussi \(T(x) \text{avec} T \in \ens(D)'\)
	\item On ne peut pas définir le produit entre deux distributions quelconques, la théorie des distributions ne s'applique qu'à des problèmes linéaires. En particulier les équations différentielles ordinaires (E.D.O) linéaire
\end{enumerate}

\paragraph{Définition 1.10.} Multiplication par une fonction \(C^{\infty}\)

Si \(f \in C^{\infty}(\ens{R} \) et \(T\in \D'\) on peut définir la distribution \(f\cdot T \in ens{D}'\) par :
\[
	\langle f\cdot T, \varphi \rangle = \langle T, f\cdot \varphi \rangle \quad \forall \varphi \in \D
\]

\paragraph{Exemple :} \(\cos \cdot \delta\)

\(cos \in C^{\infty}\) et \(\delta \in \D'\)
Pour définir \(\cos \cdot delta\) il faut trouver la valeur de \(\langle \cos \cdot \delta, \varphi \rangle\)
\begin{align*}
	\langle \cos \cdot \delta, \varphi \rangle &= \langle \delta, \cos \varphi \rangle\\
	&= (\cos \cdot \varphi)(0)\\
	&= \cos(0) \cdot \varphi(0)\\
	&= \varphi (0)\\
	&= \langle \delta, \varphi \rangle
\end{align*}

\paragraph{Exemples de distributions}
\paragraph{1. Distributions régulières}
Les fonction \(L^1_{loc}\) sont les plus utilisées :
\begin{enumerate}
	\item Fonctions constantes : \(f(x) =1 \) par exemple
	\[
		\langle T, \varphi \rangle = \int_{\ens{R}} \varphi(x) dx
	\]
	\item Fonctions indicatrices : \(f = \chi_{[a,b]}\)
	\[
		\langle T, \varphi \rangle = \int_{a}^b \varphi(x) dx
	\]
	\item Fonction Heavyside : \(H = \chi_{[0,+\infty[}\)
	\[
		\langle T, \varphi \rangle = \int_{0}^{+\infty} \varphi(x) dx
	\]
	\item Fonction signe ; \(Sgn ) = \chi_{[0,+\infty[} - \chi_{]-\infty, 0]} \)
	\[
		\langle T, \varphi \rangle = \int_{0}^{+\infty} \varphi(x) dx - \int_{-\infty}^0 \varphi(x) dx
	\]
	\item Et beaucoup d'autre fonctions : \(f = |vert \cos \vert, f = ln |vert x \vert\)
\end{enumerate}

\(f =_{pp} g\) dans \(L^1_{loc} \Leftrightarrow T_f = T_g\) dans \D'

\paragraph{2. Distributions singulières}

\paragraph{Définition 1.11. Distributions de Dirac}
On note \(\delta_a\) la distribution de dirac au point a. On notera parfois \(\delta_a = \delta(x-a)\)

\paragraph{Définition 1.12. Distribution peigne de Dirac}
	Un peigne de Dirac noté \(\dc_a\) est une somme de distribution de dirac sous la forme :
	\[
		\langle \dc_a, \varphi \rangle = \langle \sum_{n = -\infty}^{+\infty} \delta_{na}, \varphi \rangle = \sum_{n = -\infty}^{+\infty} \varphi(na)
	\]
	où a est un réel constant. Le peigne de dirac permet d'échantilloner la fonction \(\varphi\) et de faire la somme de toutes les valeurs échantillonnées.
	
\paragraph{Définition 1.13. Distribution Valeur princtipale de \(1/x\)}

On définit la distribution valeur principale de \(1/x\) notée \(Vp(\frac{1}{x}\) par :
\[
	\langle Vp(\frac{1}{x}, \varphi \rangle = \lim_{\epsilon \rightarrow 0^+} \int_{\vert x \vert > \epsilon} \dfrac{\varphi(x)}{x}dx \fphi
\]

\paragraph{Définition 1.14. Distribution Partie finie de \(1/x^2\)}

On définit la distrtibution Partie finie de \(1/x^2\) notée Pf(\(\frac{1}{x^2}\) par :
\[
	\langle Pf(\frac{1}{x^2}, \varphi \rangle = \lim_{\epsilon \rightarrow 0^+} \int_{\vert x \vert > \epsilon} \dfrac{\varphi(x)}{x^2}dx \fphi
\]

\section{Opérations et propriétés élémentaires dans \D'}

\subsection{Translation}
\paragraph{Définition 1.14.} (Rappel) Translations des fonctions

Si f est une fonction, on définit \(\tau_af\) translaté de a de la fonction f par :
\[
	\tau_af = f(x - a)
\]

Comment définir \(\tau_a T\) si \(T \in \D'\) et T quelconque ?

\[
	\langle \tau_a [f], \varphi \rangle = \langle [\tau_a f], \varphi \rangle = \int_{\ens{R}} \tau_a f(x) \varphi(x) dx
\]
\begin{align*}
	&= \int_{\ens{R}} f(x-a) \varphi(x) dx, \text{chgmt de variable } y = x - a, x = y + a, J = 1\\
	&= \int_{\ens{R}} f(y) \varphi(y+a) dy\\
	&= \int_{\ens{R}} f(y) \tau_{-a} \varphi(y) dy\\
	&= \langle [f], \tau_{-a}\varphi \rangle
\end{align*}

\begin{center}
\textbf{CM3 (2021-01-21)}
\end{center}

\paragraph{Définition 1.16.} Translation dans \D'

Si \(T \in \D'\) et \(a \in \ens{R}\), on définit \(\tau_a T \in \D'\) :
\[
	\langle \tau_a T, \varphi \rangle = \langle T, \tau_{-a} \varphi \rangle = \langle T, \varphi(x+a) \rangle
\]

\paragraph{Exemple :} \(\tau_a delta\)
\[
	\langle \delta, \varphi \rangle = \varphi(0)
\]
\begin{align*}
	\langle \tau_a \delta, \varphi \rangle &= \langle \delta, \varphi(x+a) \rangle\\
	&= \varphi (x+a)\vert_{x = 0}\\
	&= \varphi(a)\\
	&= \langle \delta_a, \varphi \rangle
\end{align*}

\subsection{Dilatation}

\paragraph{Définition 1.17.} (Rappel) Dilatation des fonctions

Si f est une fonction et \(\lambda \in \ens{R^*}\), on définit \(d_\lambda f\) dilatée de \(\lambda\) de la fonction g par :
\[
	d_\lambda f (x) = f(\dfrac{x}{\lambda}) 
\]

On établit une formule pour \(d_\lambda T, T \in \D'\), lorsque T est régulière
\begin{align*}
	\langle d_\lambda [f], \varphi \rangle &= \langle \left[d_\lambda f\right], \varphi \rangle\\
	&= \int_{\ens{R}} d_\lambda f(x) \varphi(x) dx\\
	&= \int_{\ens{R}} f(\dfrac{x}{\lambda} \varphi(x) dx\\
	& \text{On fait un changement de variable, } y= \dfrac{x}{\lambda}, x = \lambda y, J = \lambda\\
	&= \int_{\ens{R}} f(y) \varphi (\lambda y) \vert \lambda \vert dy\\
	&= \int_{\ens{R}} f(y) d_{\frac{1}{\lambda}}\varphi (y) \vert \lambda \vert dy\\
	&= \langle [f], \vert \lambda \vert d_{\frac{1}{\lambda}} \varphi \rangle
\end{align*}

\paragraph{Définition 1.18} Dilatation dans \D'

Si \(T \in \D'\) et \(\lambda \in \ens{R^*}\), on définit \(d_\lambda T \in \D'\) la dilaté de la distribution T par :
\[
	\langle d_\lambda T, \varphi \rangle = \langle [f], \vert \lambda \vert d_{\frac{1}{\lambda}} \varphi \rangle = \langle [f], \vert \lambda \vert \varphi (\lambda x) \rangle 
\]

\paragraph{Exemple : } \(\delta_0 = \delta\)
\[
	\langle \delta, \varphi \rangle = \varphi(0)
\]
On cherche \(\delta_{-3}\)
\begin{align*}
	\langle \delta_{-3}, \varphi \rangle &=_{def} \langle [f], \vert -3 \vert d_{\frac{-1}{3}} \varphi \rangle\\
	&= \langle [f], 3 \varphi(-3x) \rangle\\
	&= 3 \varphi (-3 \times 0) = 3 \varphi (0)\\
	&= 3 \langle \delta, \varphi \rangle = \langle 3\delta, \varphi \rangle 
\end{align*}

Donc \(\delta_{-3} = 3\delta\) 

\paragraph{Définition 1.19} Parité dans \D'

T est une distribution paire \(\Leftrightarrow d_{-1} T = T\)
T est une distribution impaire \(Leftrightarrow d_{-1} T = -T\)

\paragraph{Exemple} distribution régulière impaire

Soit [x] régulière associé à la fonction (\(x \rightarrow x\)). Cette fonction est impaire. Montrons que la distribution [x] est impaire dans \D'

Remarque \([x] \in \D'\) existe.

\begin{align*}
	\langle d_{-1} [x], \varphi \rangle &= \langle [x],\vert -1 \vert d_{-1} \varphi \rangle = \langle [x], \varphi(-x) \rangle\\
	&= \int_{\ens{R}} x \varphi(-x) dx\\
	&= \int_{\ens{R}} -y \varphi(y) dy\\
	&= - \int_{\ens{R}} y \varphi(y) dy = - \langle [x], \varphi \rangle = \langle -[x], \varphi \rangle\\
\end{align*}

On retiendra qu'une distribution régulière associé à une fonction paire (ou impaire) est paire (ou impaire).

\paragraph{Définition 1.20.} Périodicité dans \D'

Si \(T \in \D'\), \(T\) est dite périodique de période a si :
\[
	\tau_a T = T
\]

Toutes distribution régulière associé à une fonction périodique est périodique.

\subsection{Convergence dans \D'}

\paragraph{Définition 1.21.} Convergence dans \D'

Soit \((T_n)_{n\in\ens{N}}\) une suite de distribution dans \D'.

On dit que \(T_n\) converge vers T dans \D' et on notera \(\underset{n \rightarrow +\infty}{\lim} T_n = T\) ou \(T_n \underset{n \rightarrow +\infty}{\rightarrow} T\) si :
\[
	\langle T_n, \varphi \rangle \underset{n \rightarrow +\infty}{\rightarrow} \langle T, \varphi \rangle
\]

\paragraph{Exemple 1} Dirac peut s'obtenir par la limite d'une suite de distribution :
\[
	f_n(t) = \left\{
    \begin{array}{ll}
        n & \mbox{si } t \in [-\frac{1}{2n}, \frac{1}{2n}] \\
        0 & \mbox{sinon.}
    \end{array}
\right.
\]
Mais au sens des distributions on va montrer que \([f_n] \underset{n \rightarrow +\infty}{\rightarrow} \delta\)
\[
	\langle [f_n], \varphi \rangle = \int_{\ens{R}} f_n(x) \varphi(x) dx = n \int_{-\frac{1}{2n}}^{\frac{1}{2n}} \varphi(x) dx
\]

\[
	\langle [f_n], \varphi \rangle = n \left[ \Phi \right]_{-\frac{1}{2n}}^{\frac{1}{2n}} = \underset{n \rightarrow +\infty}{\rightarrow} \Phi' (0)
\]

\begin{center}
\textbf{CM4 (2021-01-2x)}
\end{center}

\end{document}
